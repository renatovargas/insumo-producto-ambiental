% Options for packages loaded elsewhere
\PassOptionsToPackage{unicode}{hyperref}
\PassOptionsToPackage{hyphens}{url}
%
\documentclass[
]{article}
\usepackage{amsmath,amssymb}
\usepackage{lmodern}
\usepackage{ifxetex,ifluatex}
\ifnum 0\ifxetex 1\fi\ifluatex 1\fi=0 % if pdftex
  \usepackage[T1]{fontenc}
  \usepackage[utf8]{inputenc}
  \usepackage{textcomp} % provide euro and other symbols
\else % if luatex or xetex
  \usepackage{unicode-math}
  \defaultfontfeatures{Scale=MatchLowercase}
  \defaultfontfeatures[\rmfamily]{Ligatures=TeX,Scale=1}
\fi
% Use upquote if available, for straight quotes in verbatim environments
\IfFileExists{upquote.sty}{\usepackage{upquote}}{}
\IfFileExists{microtype.sty}{% use microtype if available
  \usepackage[]{microtype}
  \UseMicrotypeSet[protrusion]{basicmath} % disable protrusion for tt fonts
}{}
\makeatletter
\@ifundefined{KOMAClassName}{% if non-KOMA class
  \IfFileExists{parskip.sty}{%
    \usepackage{parskip}
  }{% else
    \setlength{\parindent}{0pt}
    \setlength{\parskip}{6pt plus 2pt minus 1pt}}
}{% if KOMA class
  \KOMAoptions{parskip=half}}
\makeatother
\usepackage{xcolor}
\IfFileExists{xurl.sty}{\usepackage{xurl}}{} % add URL line breaks if available
\IfFileExists{bookmark.sty}{\usepackage{bookmark}}{\usepackage{hyperref}}
\hypersetup{
  pdftitle={modulo\_03.R},
  pdfauthor={renato},
  hidelinks,
  pdfcreator={LaTeX via pandoc}}
\urlstyle{same} % disable monospaced font for URLs
\usepackage[margin=1in]{geometry}
\usepackage{color}
\usepackage{fancyvrb}
\newcommand{\VerbBar}{|}
\newcommand{\VERB}{\Verb[commandchars=\\\{\}]}
\DefineVerbatimEnvironment{Highlighting}{Verbatim}{commandchars=\\\{\}}
% Add ',fontsize=\small' for more characters per line
\usepackage{framed}
\definecolor{shadecolor}{RGB}{248,248,248}
\newenvironment{Shaded}{\begin{snugshade}}{\end{snugshade}}
\newcommand{\AlertTok}[1]{\textcolor[rgb]{0.94,0.16,0.16}{#1}}
\newcommand{\AnnotationTok}[1]{\textcolor[rgb]{0.56,0.35,0.01}{\textbf{\textit{#1}}}}
\newcommand{\AttributeTok}[1]{\textcolor[rgb]{0.77,0.63,0.00}{#1}}
\newcommand{\BaseNTok}[1]{\textcolor[rgb]{0.00,0.00,0.81}{#1}}
\newcommand{\BuiltInTok}[1]{#1}
\newcommand{\CharTok}[1]{\textcolor[rgb]{0.31,0.60,0.02}{#1}}
\newcommand{\CommentTok}[1]{\textcolor[rgb]{0.56,0.35,0.01}{\textit{#1}}}
\newcommand{\CommentVarTok}[1]{\textcolor[rgb]{0.56,0.35,0.01}{\textbf{\textit{#1}}}}
\newcommand{\ConstantTok}[1]{\textcolor[rgb]{0.00,0.00,0.00}{#1}}
\newcommand{\ControlFlowTok}[1]{\textcolor[rgb]{0.13,0.29,0.53}{\textbf{#1}}}
\newcommand{\DataTypeTok}[1]{\textcolor[rgb]{0.13,0.29,0.53}{#1}}
\newcommand{\DecValTok}[1]{\textcolor[rgb]{0.00,0.00,0.81}{#1}}
\newcommand{\DocumentationTok}[1]{\textcolor[rgb]{0.56,0.35,0.01}{\textbf{\textit{#1}}}}
\newcommand{\ErrorTok}[1]{\textcolor[rgb]{0.64,0.00,0.00}{\textbf{#1}}}
\newcommand{\ExtensionTok}[1]{#1}
\newcommand{\FloatTok}[1]{\textcolor[rgb]{0.00,0.00,0.81}{#1}}
\newcommand{\FunctionTok}[1]{\textcolor[rgb]{0.00,0.00,0.00}{#1}}
\newcommand{\ImportTok}[1]{#1}
\newcommand{\InformationTok}[1]{\textcolor[rgb]{0.56,0.35,0.01}{\textbf{\textit{#1}}}}
\newcommand{\KeywordTok}[1]{\textcolor[rgb]{0.13,0.29,0.53}{\textbf{#1}}}
\newcommand{\NormalTok}[1]{#1}
\newcommand{\OperatorTok}[1]{\textcolor[rgb]{0.81,0.36,0.00}{\textbf{#1}}}
\newcommand{\OtherTok}[1]{\textcolor[rgb]{0.56,0.35,0.01}{#1}}
\newcommand{\PreprocessorTok}[1]{\textcolor[rgb]{0.56,0.35,0.01}{\textit{#1}}}
\newcommand{\RegionMarkerTok}[1]{#1}
\newcommand{\SpecialCharTok}[1]{\textcolor[rgb]{0.00,0.00,0.00}{#1}}
\newcommand{\SpecialStringTok}[1]{\textcolor[rgb]{0.31,0.60,0.02}{#1}}
\newcommand{\StringTok}[1]{\textcolor[rgb]{0.31,0.60,0.02}{#1}}
\newcommand{\VariableTok}[1]{\textcolor[rgb]{0.00,0.00,0.00}{#1}}
\newcommand{\VerbatimStringTok}[1]{\textcolor[rgb]{0.31,0.60,0.02}{#1}}
\newcommand{\WarningTok}[1]{\textcolor[rgb]{0.56,0.35,0.01}{\textbf{\textit{#1}}}}
\usepackage{graphicx}
\makeatletter
\def\maxwidth{\ifdim\Gin@nat@width>\linewidth\linewidth\else\Gin@nat@width\fi}
\def\maxheight{\ifdim\Gin@nat@height>\textheight\textheight\else\Gin@nat@height\fi}
\makeatother
% Scale images if necessary, so that they will not overflow the page
% margins by default, and it is still possible to overwrite the defaults
% using explicit options in \includegraphics[width, height, ...]{}
\setkeys{Gin}{width=\maxwidth,height=\maxheight,keepaspectratio}
% Set default figure placement to htbp
\makeatletter
\def\fps@figure{htbp}
\makeatother
\setlength{\emergencystretch}{3em} % prevent overfull lines
\providecommand{\tightlist}{%
  \setlength{\itemsep}{0pt}\setlength{\parskip}{0pt}}
\setcounter{secnumdepth}{-\maxdimen} % remove section numbering
\ifluatex
  \usepackage{selnolig}  % disable illegal ligatures
\fi

\title{modulo\_03.R}
\author{renato}
\date{2021-11-17}

\begin{document}
\maketitle

\begin{Shaded}
\begin{Highlighting}[]
\CommentTok{\# Insumo{-}Producto Ambientalmente Extendido}
\CommentTok{\# Facilitado por Renato Vargas}
\CommentTok{\# Modulo 03 {-} Matriz de Insumo Producto y }
\CommentTok{\# Cuentas Ambientales de Costa Rica}
\CommentTok{\# Publicadas por el Banco Central de Costa Rica}
\CommentTok{\# Año 2017}

\CommentTok{\# Preámbulo}
\FunctionTok{library}\NormalTok{(openxlsx)}
\end{Highlighting}
\end{Shaded}

\begin{verbatim}
## Warning: package 'openxlsx' was built under R version 4.0.5
\end{verbatim}

\begin{Shaded}
\begin{Highlighting}[]
\FunctionTok{library}\NormalTok{(reshape2)}
\FunctionTok{library}\NormalTok{(Matrix.utils)}
\end{Highlighting}
\end{Shaded}

\begin{verbatim}
## Warning: package 'Matrix.utils' was built under R version 4.0.5
\end{verbatim}

\begin{verbatim}
## Loading required package: Matrix
\end{verbatim}

\begin{Shaded}
\begin{Highlighting}[]
\FunctionTok{rm}\NormalTok{(  }\AttributeTok{list =} \FunctionTok{ls}\NormalTok{()  )}


\CommentTok{\# Directorio de trabajo (ruta a los datos con "/" en vez de "\textbackslash{}")}
\NormalTok{wd }\OtherTok{\textless{}{-}} \FunctionTok{c}\NormalTok{(}\StringTok{"D:/github/insumo{-}producto{-}ambiental/datos"}\NormalTok{)}
\FunctionTok{setwd}\NormalTok{(wd)}
\FunctionTok{getwd}\NormalTok{()}
\end{Highlighting}
\end{Shaded}

\begin{verbatim}
## [1] "D:/github/insumo-producto-ambiental/datos"
\end{verbatim}

\begin{Shaded}
\begin{Highlighting}[]
\CommentTok{\# Consumo intermedio}
\NormalTok{Z\_cruda }\OtherTok{\textless{}{-}} \FunctionTok{as.matrix}\NormalTok{(}\FunctionTok{read.xlsx}\NormalTok{(}\StringTok{"MIP{-}AE{-}AE{-}017{-}CR.xlsx"}\NormalTok{, }
                               \AttributeTok{sheet =} \StringTok{"MIP 2017"}\NormalTok{, }
                               \AttributeTok{rows=} \FunctionTok{c}\NormalTok{(}\DecValTok{12}\SpecialCharTok{:}\DecValTok{264}\NormalTok{), }
                               \AttributeTok{cols =} \FunctionTok{c}\NormalTok{(}\DecValTok{4}\SpecialCharTok{:}\DecValTok{256}\NormalTok{), }
                               \AttributeTok{skipEmptyRows =} \ConstantTok{FALSE}\NormalTok{, }
                               \AttributeTok{colNames =} \ConstantTok{FALSE}\NormalTok{, }
                               \AttributeTok{rowNames =} \ConstantTok{FALSE}\NormalTok{)}
\NormalTok{                     )  }\CommentTok{\# \textless{}{-}{-} Fin del paréntesis}

\NormalTok{nombres }\OtherTok{\textless{}{-}} \FunctionTok{read.xlsx}\NormalTok{(}\StringTok{"MIP{-}AE{-}AE{-}017{-}CR.xlsx"}\NormalTok{, }
                               \AttributeTok{sheet =} \StringTok{"MIP 2017"}\NormalTok{, }
                               \AttributeTok{rows=} \FunctionTok{c}\NormalTok{(}\DecValTok{12}\SpecialCharTok{:}\DecValTok{264}\NormalTok{), }
                               \AttributeTok{cols =} \DecValTok{1}\NormalTok{, }
                               \AttributeTok{skipEmptyRows =} \ConstantTok{FALSE}\NormalTok{, }
                               \AttributeTok{colNames =} \ConstantTok{FALSE}\NormalTok{, }
                               \AttributeTok{rowNames =} \ConstantTok{FALSE}
\NormalTok{)  }\CommentTok{\# \textless{}{-}{-} Fin del paréntesis}


\CommentTok{\# Agregamos nuestra matriz para tener un valor por actividad}

\CommentTok{\# Nombramos nuestra matriz Z}
\FunctionTok{colnames}\NormalTok{(Z\_cruda) }\OtherTok{\textless{}{-}} \FunctionTok{as.vector}\NormalTok{(nombres}\SpecialCharTok{$}\NormalTok{X1)}
\FunctionTok{rownames}\NormalTok{(Z\_cruda) }\OtherTok{\textless{}{-}} \FunctionTok{as.vector}\NormalTok{(nombres}\SpecialCharTok{$}\NormalTok{X1)}

\NormalTok{Z }\OtherTok{\textless{}{-}} \FunctionTok{aggregate.Matrix}\NormalTok{(Z\_cruda, }\FunctionTok{as.factor}\NormalTok{(nombres}\SpecialCharTok{$}\NormalTok{X1),}\AttributeTok{fun =} \StringTok{"sum"}\NormalTok{)}
\NormalTok{Z }\OtherTok{\textless{}{-}} \FunctionTok{t}\NormalTok{(}\FunctionTok{aggregate.Matrix}\NormalTok{(}\FunctionTok{t}\NormalTok{(Z), }\FunctionTok{as.factor}\NormalTok{(nombres}\SpecialCharTok{$}\NormalTok{X1),}\AttributeTok{fun =} \StringTok{"sum"}\NormalTok{))}
\NormalTok{Z }\OtherTok{\textless{}{-}} \FunctionTok{as.matrix}\NormalTok{(Z)}

\FunctionTok{dim}\NormalTok{(Z)}
\end{Highlighting}
\end{Shaded}

\begin{verbatim}
## [1] 136 136
\end{verbatim}

\begin{Shaded}
\begin{Highlighting}[]
\CommentTok{\# Demanda Final}
\NormalTok{DF\_cruda }\OtherTok{\textless{}{-}} \FunctionTok{as.matrix}\NormalTok{(}\FunctionTok{read.xlsx}\NormalTok{(}\StringTok{"MIP{-}AE{-}AE{-}017{-}CR.xlsx"}\NormalTok{, }
                               \AttributeTok{sheet =} \StringTok{"MIP 2017"}\NormalTok{, }
                               \AttributeTok{rows=} \FunctionTok{c}\NormalTok{(}\DecValTok{12}\SpecialCharTok{:}\DecValTok{264}\NormalTok{), }
                               \AttributeTok{cols =} \FunctionTok{c}\NormalTok{(}\DecValTok{258}\SpecialCharTok{:}\DecValTok{262}\NormalTok{), }
                               \AttributeTok{skipEmptyRows =} \ConstantTok{FALSE}\NormalTok{, }
                               \AttributeTok{colNames =} \ConstantTok{FALSE}\NormalTok{, }
                               \AttributeTok{rowNames =} \ConstantTok{FALSE}\NormalTok{)}
\NormalTok{                     )  }\CommentTok{\# \textless{}{-}{-} Fin del paréntesis}


\CommentTok{\# Lo mismo para la demanda final}
\NormalTok{codsDF }\OtherTok{\textless{}{-}} \FunctionTok{as.matrix}\NormalTok{(}
  \FunctionTok{t}\NormalTok{(}
    \FunctionTok{read.xlsx}\NormalTok{(}
      \StringTok{"MIP{-}AE{-}AE{-}017{-}CR.xlsx"}\NormalTok{,}
      \AttributeTok{sheet =} \StringTok{"MIP 2017"}\NormalTok{, }
      \AttributeTok{rows=} \FunctionTok{c}\NormalTok{(}\DecValTok{9}\SpecialCharTok{:}\DecValTok{10}\NormalTok{), }
      \AttributeTok{cols =} \FunctionTok{c}\NormalTok{(}\DecValTok{258}\SpecialCharTok{:}\DecValTok{262}\NormalTok{), }
      \AttributeTok{skipEmptyRows =} \ConstantTok{FALSE}\NormalTok{,}
      \AttributeTok{colNames =} \ConstantTok{FALSE}\NormalTok{, }
      \AttributeTok{rowNames =} \ConstantTok{FALSE}
\NormalTok{  ))}
\NormalTok{)}\CommentTok{\# \textless{}{-}{-} Fin del paréntesis}

\CommentTok{\# Y nuestra matriz de demanda final DF}
\FunctionTok{colnames}\NormalTok{(DF\_cruda) }\OtherTok{\textless{}{-}} \FunctionTok{as.vector}\NormalTok{(codsDF[,}\DecValTok{1}\NormalTok{])}
\FunctionTok{rownames}\NormalTok{(DF\_cruda) }\OtherTok{\textless{}{-}} \FunctionTok{as.vector}\NormalTok{(}\FunctionTok{rownames}\NormalTok{(Z\_cruda))}

\CommentTok{\# Y agregamos las filas que se repiten}
\NormalTok{DF }\OtherTok{\textless{}{-}} \FunctionTok{aggregate.Matrix}\NormalTok{(DF\_cruda, }\FunctionTok{as.factor}\NormalTok{(}\FunctionTok{rownames}\NormalTok{(DF\_cruda)),}\AttributeTok{fun =} \StringTok{"sum"}\NormalTok{)}
\NormalTok{DF }\OtherTok{\textless{}{-}} \FunctionTok{as.matrix}\NormalTok{(DF)}

\FunctionTok{dim}\NormalTok{(DF)}
\end{Highlighting}
\end{Shaded}

\begin{verbatim}
## [1] 136   5
\end{verbatim}

\begin{Shaded}
\begin{Highlighting}[]
\CommentTok{\# =============================================================================}
\CommentTok{\# Cuenta de energía}

\CommentTok{\# Importamos los datos crudos}
\NormalTok{E\_cruda }\OtherTok{\textless{}{-}} \FunctionTok{as.matrix}\NormalTok{(}\FunctionTok{read.xlsx}\NormalTok{(}\StringTok{"COUF{-}2017.xlsx"}\NormalTok{, }
                               \AttributeTok{sheet =} \StringTok{"COUF{-}E 2017"}\NormalTok{, }
                               \AttributeTok{rows=} \FunctionTok{c}\NormalTok{(}\DecValTok{127}\SpecialCharTok{:}\DecValTok{146}\NormalTok{), }
                               \AttributeTok{cols =} \FunctionTok{c}\NormalTok{(}\DecValTok{3}\SpecialCharTok{:}\DecValTok{169}\NormalTok{), }
                               \AttributeTok{skipEmptyRows =} \ConstantTok{FALSE}\NormalTok{, }
                               \AttributeTok{colNames =} \ConstantTok{FALSE}\NormalTok{, }
                               \AttributeTok{rowNames =} \ConstantTok{TRUE} \CommentTok{\# Sí hay nombres de fila}
\NormalTok{                               )}
\NormalTok{)  }\CommentTok{\# \textless{}{-}{-} Fin del paréntesis}

\CommentTok{\# Extraemos los nombres de columna}
\NormalTok{nombres\_e }\OtherTok{\textless{}{-}} \FunctionTok{t}\NormalTok{(}\FunctionTok{read.xlsx}\NormalTok{(}\StringTok{"COUF{-}2017.xlsx"}\NormalTok{, }
                               \AttributeTok{sheet =} \StringTok{"COUF{-}E 2017"}\NormalTok{, }
                               \AttributeTok{rows=} \FunctionTok{c}\NormalTok{(}\DecValTok{16}\NormalTok{), }
                               \AttributeTok{cols =} \FunctionTok{c}\NormalTok{(}\DecValTok{4}\SpecialCharTok{:}\DecValTok{169}\NormalTok{), }
                               \AttributeTok{skipEmptyRows =} \ConstantTok{FALSE}\NormalTok{, }
                               \AttributeTok{colNames =} \ConstantTok{FALSE}\NormalTok{, }
                               \AttributeTok{rowNames =} \ConstantTok{FALSE}\NormalTok{)}
\NormalTok{                     )  }\CommentTok{\# \textless{}{-}{-} Fin del paréntesis}

\CommentTok{\# Y nombramos las columnas de nuestra matriz de usos energéticos}
\FunctionTok{colnames}\NormalTok{(E\_cruda) }\OtherTok{\textless{}{-}} \FunctionTok{c}\NormalTok{(nombres\_e[,}\DecValTok{1}\NormalTok{])}

\CommentTok{\# Las dimensiones de E\_cruda son mayores a las de Z}
\CommentTok{\# porque hay agregaciones por grupos de sectores y}
\CommentTok{\# hay sectores desagregados a mayor detalle.}

\FunctionTok{dim}\NormalTok{(E\_cruda)}
\end{Highlighting}
\end{Shaded}

\begin{verbatim}
## [1]  20 166
\end{verbatim}

\begin{Shaded}
\begin{Highlighting}[]
\CommentTok{\# Identificamos las posiciones que son sumas de sectores}
\CommentTok{\# Nótese que dejamos dos sumas dentro que no tienen detalle}
\CommentTok{\# correspondientes a AE082 (Electricidad) y AE144 (hogares como empl.)}

\NormalTok{posGruposEnergia }\OtherTok{\textless{}{-}} \FunctionTok{c}\NormalTok{(}\DecValTok{1}\NormalTok{,}\DecValTok{31}\NormalTok{,}\DecValTok{35}\NormalTok{,}\DecValTok{78}\NormalTok{,}\DecValTok{82}\NormalTok{,}\DecValTok{94}\NormalTok{,}\DecValTok{97}\NormalTok{,}\DecValTok{106}\NormalTok{,}\DecValTok{110}\NormalTok{,}\DecValTok{113}\NormalTok{,}
                      \DecValTok{119}\NormalTok{,}\DecValTok{122}\NormalTok{,}\DecValTok{133}\NormalTok{,}\DecValTok{143}\NormalTok{,}\DecValTok{147}\NormalTok{,}\DecValTok{150}\NormalTok{,}\DecValTok{153}\NormalTok{,}\DecValTok{158}\NormalTok{)}

\CommentTok{\# Extraemos solo los sectores (nótese el "{-}" antes de posGruposEnergia)}
\NormalTok{E\_cruda }\OtherTok{\textless{}{-}}\NormalTok{ E\_cruda[ , }\SpecialCharTok{{-}}\NormalTok{posGruposEnergia]}

\CommentTok{\# Utilizando la función substr() extraemos los primeros 5 digitos de}
\CommentTok{\# la nomenclatura para poder agregar por actividades que comparten}
\CommentTok{\# esos mismos.}
\FunctionTok{colnames}\NormalTok{(E\_cruda) }\OtherTok{\textless{}{-}} \FunctionTok{substr}\NormalTok{(}\FunctionTok{colnames}\NormalTok{(E\_cruda), }\AttributeTok{start =} \DecValTok{1}\NormalTok{, }\AttributeTok{stop =} \DecValTok{5}\NormalTok{)}

\CommentTok{\# Y agregamos utilizando el mismo procedimiento que anteriormente.}
\NormalTok{E }\OtherTok{\textless{}{-}} \FunctionTok{as.matrix}\NormalTok{(}\FunctionTok{t}\NormalTok{(}\FunctionTok{aggregate.Matrix}\NormalTok{(}\FunctionTok{t}\NormalTok{(E\_cruda), }\FunctionTok{colnames}\NormalTok{(E\_cruda),}\AttributeTok{fun =} \StringTok{"sum"}\NormalTok{)))}
\NormalTok{E }\OtherTok{\textless{}{-}} \FunctionTok{as.matrix}\NormalTok{(E)}
\CommentTok{\# Y chequeamos que nuestras dimensiones sean iguales a las columnas}
\CommentTok{\# de Z}
\FunctionTok{dim}\NormalTok{(E)}
\end{Highlighting}
\end{Shaded}

\begin{verbatim}
## [1]  20 136
\end{verbatim}

\begin{Shaded}
\begin{Highlighting}[]
\CommentTok{\# Para ser congruentes con Z, renombramos las columnas con los nombres}
\CommentTok{\# completos de Z}
\FunctionTok{colnames}\NormalTok{(E) }\OtherTok{\textless{}{-}} \FunctionTok{colnames}\NormalTok{(Z)}

\CommentTok{\# Hacemos limpieza}
\FunctionTok{rm}\NormalTok{(Z\_cruda,DF\_cruda,E\_cruda, nombres\_e)}

\CommentTok{\# =============================================================================}
\DocumentationTok{\#\# Cuenta de energía 2018 (Preliminar No{-}Citar)}

\CommentTok{\# Importamos los datos crudos}
\NormalTok{E\_cruda }\OtherTok{\textless{}{-}} \FunctionTok{as.matrix}\NormalTok{(}\FunctionTok{read.xlsx}\NormalTok{(}\StringTok{"COUF{-}2018.xlsx"}\NormalTok{, }
                               \AttributeTok{sheet =} \StringTok{"COUF{-}E 2018"}\NormalTok{, }
                               \AttributeTok{rows=} \FunctionTok{c}\NormalTok{(}\DecValTok{126}\SpecialCharTok{:}\DecValTok{145}\NormalTok{), }
                               \AttributeTok{cols =} \FunctionTok{c}\NormalTok{(}\DecValTok{3}\SpecialCharTok{:}\DecValTok{170}\NormalTok{), }
                               \AttributeTok{skipEmptyRows =} \ConstantTok{FALSE}\NormalTok{, }
                               \AttributeTok{colNames =} \ConstantTok{FALSE}\NormalTok{, }
                               \AttributeTok{rowNames =} \ConstantTok{TRUE} \CommentTok{\# Sí hay nombres de fila}
\NormalTok{                               )}
\NormalTok{)  }\CommentTok{\# \textless{}{-}{-} Fin del paréntesis}

\CommentTok{\# Extraemos los nombres de columna}
\NormalTok{nombres\_e }\OtherTok{\textless{}{-}} \FunctionTok{t}\NormalTok{(}\FunctionTok{read.xlsx}\NormalTok{(}\StringTok{"COUF{-}2018.xlsx"}\NormalTok{, }
                         \AttributeTok{sheet =} \StringTok{"COUF{-}E 2018"}\NormalTok{, }
                         \AttributeTok{rows=} \FunctionTok{c}\NormalTok{(}\DecValTok{15}\NormalTok{), }
                         \AttributeTok{cols =} \FunctionTok{c}\NormalTok{(}\DecValTok{4}\SpecialCharTok{:}\DecValTok{170}\NormalTok{), }
                         \AttributeTok{skipEmptyRows =} \ConstantTok{FALSE}\NormalTok{, }
                         \AttributeTok{colNames =} \ConstantTok{FALSE}\NormalTok{, }
                         \AttributeTok{rowNames =} \ConstantTok{FALSE}\NormalTok{)}
\NormalTok{)  }\CommentTok{\# \textless{}{-}{-} Fin del paréntesis}

\CommentTok{\# Y nombramos las columnas de nuestra matriz de usos energéticos}
\FunctionTok{colnames}\NormalTok{(E\_cruda) }\OtherTok{\textless{}{-}} \FunctionTok{c}\NormalTok{(nombres\_e[,}\DecValTok{1}\NormalTok{])}

\CommentTok{\# Las dimensiones de E\_cruda son mayores a las de Z}
\CommentTok{\# porque hay agregaciones por grupos de sectores y}
\CommentTok{\# hay sectores desagregados a mayor detalle.}

\FunctionTok{dim}\NormalTok{(E\_cruda)}
\end{Highlighting}
\end{Shaded}

\begin{verbatim}
## [1]  20 167
\end{verbatim}

\begin{Shaded}
\begin{Highlighting}[]
\CommentTok{\# Identificamos las posiciones que son sumas de sectores}
\CommentTok{\# Nótese que dejamos dos sumas dentro que no tienen detalle}
\CommentTok{\# correspondientes a AE082 (Electricidad) y AE144 (hogares como empl.)}

\CommentTok{\# Notar que respecto de 2017 lo siguiente cambia a partir de 106 (107)}

\NormalTok{posGruposEnergia }\OtherTok{\textless{}{-}}\FunctionTok{c}\NormalTok{(}\DecValTok{1}\NormalTok{,}\DecValTok{31}\NormalTok{,}\DecValTok{35}\NormalTok{,}\DecValTok{78}\NormalTok{,}\DecValTok{82}\NormalTok{,}\DecValTok{94}\NormalTok{,}\DecValTok{97}\NormalTok{,}\DecValTok{107}\NormalTok{,}\DecValTok{111}\NormalTok{,}\DecValTok{114}\NormalTok{,}
                      \DecValTok{120}\NormalTok{,}\DecValTok{123}\NormalTok{,}\DecValTok{134}\NormalTok{,}\DecValTok{144}\NormalTok{,}\DecValTok{148}\NormalTok{,}\DecValTok{151}\NormalTok{,}\DecValTok{154}\NormalTok{,}\DecValTok{159}\NormalTok{)}

\CommentTok{\# Extraemos solo los sectores (nótese el "{-}" antes de posGruposEnergia)}
\NormalTok{E\_cruda }\OtherTok{\textless{}{-}}\NormalTok{ E\_cruda[ , }\SpecialCharTok{{-}}\NormalTok{posGruposEnergia]}

\CommentTok{\# Utilizando la función substr() extraemos los primeros 5 digitos de}
\CommentTok{\# la nomenclatura para poder agregar por actividades que comparten}
\CommentTok{\# esos mismos.}
\FunctionTok{colnames}\NormalTok{(E\_cruda) }\OtherTok{\textless{}{-}} \FunctionTok{substr}\NormalTok{(}\FunctionTok{colnames}\NormalTok{(E\_cruda), }\AttributeTok{start =} \DecValTok{1}\NormalTok{, }\AttributeTok{stop =} \DecValTok{5}\NormalTok{)}

\CommentTok{\# Y agregamos utilizando el mismo procedimiento que anteriormente.}
\NormalTok{E18 }\OtherTok{\textless{}{-}} \FunctionTok{as.matrix}\NormalTok{(}\FunctionTok{t}\NormalTok{(}\FunctionTok{aggregate.Matrix}\NormalTok{(}\FunctionTok{t}\NormalTok{(E\_cruda), }\FunctionTok{colnames}\NormalTok{(E\_cruda),}\AttributeTok{fun =} \StringTok{"sum"}\NormalTok{)))}
\NormalTok{E18 }\OtherTok{\textless{}{-}} \FunctionTok{as.matrix}\NormalTok{(E18)}
\CommentTok{\# Y chequeamos que nuestras dimensiones sean iguales a las columnas}
\CommentTok{\# de Z}
\FunctionTok{dim}\NormalTok{(E18)}
\end{Highlighting}
\end{Shaded}

\begin{verbatim}
## [1]  20 136
\end{verbatim}

\begin{Shaded}
\begin{Highlighting}[]
\CommentTok{\# Para ser congruentes con Z, renombramos las columnas con los nombres}
\CommentTok{\# completos de Z}
\FunctionTok{colnames}\NormalTok{(E18) }\OtherTok{\textless{}{-}} \FunctionTok{colnames}\NormalTok{(Z)}

\CommentTok{\# Hacemos limpieza}
\FunctionTok{rm}\NormalTok{(E\_cruda)}

\NormalTok{moltenE18 }\OtherTok{\textless{}{-}} \FunctionTok{as.matrix}\NormalTok{(}\FunctionTok{melt}\NormalTok{(E18))}

\CommentTok{\# Exportamos a Excel}
\FunctionTok{write.xlsx}\NormalTok{( }\FunctionTok{as.data.frame}\NormalTok{(moltenE18) , }
            \StringTok{"CuentaEnergiaBD\_2018.xlsx"}\NormalTok{,}
            \AttributeTok{sheetName=} \StringTok{"datos"}\NormalTok{,}
            \AttributeTok{startRow =} \DecValTok{5}\NormalTok{,}
            \AttributeTok{startCol =} \DecValTok{1}\NormalTok{,}
            \AttributeTok{asTable =} \ConstantTok{FALSE}\NormalTok{, }
            \AttributeTok{colNames =} \ConstantTok{TRUE}\NormalTok{, }
            \AttributeTok{rowNames =} \ConstantTok{TRUE}\NormalTok{, }
            \AttributeTok{overwrite =} \ConstantTok{TRUE}
\NormalTok{)}

\CommentTok{\# =============================================================================}
\CommentTok{\# Modelo de insumo producto}

\CommentTok{\# Producción}
\NormalTok{x }\OtherTok{\textless{}{-}} \FunctionTok{as.vector}\NormalTok{(}\FunctionTok{rowSums}\NormalTok{(Z) }\SpecialCharTok{+} \FunctionTok{rowSums}\NormalTok{(DF))}

\CommentTok{\# Demanda final}
\NormalTok{f }\OtherTok{\textless{}{-}} \FunctionTok{as.vector}\NormalTok{(}\FunctionTok{rowSums}\NormalTok{(DF))}

\CommentTok{\# x sombrero}
\NormalTok{xhat }\OtherTok{\textless{}{-}} \FunctionTok{diag}\NormalTok{(x)}
\NormalTok{xhat\_inv }\OtherTok{\textless{}{-}} \FunctionTok{solve}\NormalTok{(xhat)}

\CommentTok{\# Matriz de coeficientes técnicos}
\NormalTok{A }\OtherTok{\textless{}{-}}\NormalTok{ Z }\SpecialCharTok{\%*\%} \FunctionTok{solve}\NormalTok{( xhat )}

\CommentTok{\# Matriz identidad}
\NormalTok{I }\OtherTok{\textless{}{-}} \FunctionTok{diag}\NormalTok{( }\FunctionTok{dim}\NormalTok{(A)[}\DecValTok{1}\NormalTok{])}

\CommentTok{\# Matriz de Leontief}
\NormalTok{L }\OtherTok{\textless{}{-}} \FunctionTok{solve}\NormalTok{(I }\SpecialCharTok{{-}}\NormalTok{ A )}

\CommentTok{\# Coeficientes de uso de cada energético por unidad de producto}
\NormalTok{EC }\OtherTok{\textless{}{-}}\NormalTok{ E }\SpecialCharTok{\%*\%} \FunctionTok{solve}\NormalTok{(xhat)}
\FunctionTok{colnames}\NormalTok{(EC) }\OtherTok{\textless{}{-}} \FunctionTok{colnames}\NormalTok{(Z)}

\CommentTok{\# Nueva demanda final}
\NormalTok{f1 }\OtherTok{\textless{}{-}}\NormalTok{ f}
\NormalTok{f1[}\DecValTok{80}\NormalTok{] }\OtherTok{\textless{}{-}}\NormalTok{ f1[}\DecValTok{80}\NormalTok{] }\SpecialCharTok{*}\FloatTok{1.20}

\CommentTok{\# Cálculo de nuevas demandas de energía por energético}
\NormalTok{EC }\SpecialCharTok{\%*\%}\NormalTok{ L }\SpecialCharTok{\%*\%}\NormalTok{ f1}
\end{Highlighting}
\end{Shaded}

\begin{verbatim}
##                                  [,1]
## Petróleo                     0.000000
## Carbón Mineral             217.854446
## Leña                      6155.246485
## Bagazo                    9361.769370
## Cascarilla de café         400.455355
## Otros residuos vegetales  3336.413576
## Biogás                       4.442436
## Coque                     3855.035343
## Carbón Vegetal               0.000000
## Gas licuado de petróleo   5048.034741
## Gasolina regular          5510.603805
## Gasolina super            4471.370715
## Av gas                      52.332118
## Kerosene                   232.172537
## Jet fuel                  1112.688822
## Diésel                   38229.913051
## Gasóleo                    426.222919
## IFO 380                      0.000000
## Fuel oil                  5059.333812
## Energía Eléctrica        22498.552276
\end{verbatim}

\begin{Shaded}
\begin{Highlighting}[]
\CommentTok{\# Diferencias}
\NormalTok{deltaE }\OtherTok{\textless{}{-}} \FunctionTok{cbind}\NormalTok{( }\FunctionTok{rowSums}\NormalTok{(E), }
\NormalTok{       (EC }\SpecialCharTok{\%*\%}\NormalTok{ L }\SpecialCharTok{\%*\%}\NormalTok{ f1), }
\NormalTok{       (EC }\SpecialCharTok{\%*\%}\NormalTok{ L }\SpecialCharTok{\%*\%}\NormalTok{ f1)}\SpecialCharTok{{-}} \FunctionTok{rowSums}\NormalTok{(E), }
\NormalTok{       ((EC }\SpecialCharTok{\%*\%}\NormalTok{ L }\SpecialCharTok{\%*\%}\NormalTok{ f1)}\SpecialCharTok{{-}} \FunctionTok{rowSums}\NormalTok{(E))}\SpecialCharTok{*}\DecValTok{100}\SpecialCharTok{/} \FunctionTok{rowSums}\NormalTok{(E) }
\NormalTok{       ) }\CommentTok{\# \textless{}{-}{-} fin del paréntesis}
\FunctionTok{colnames}\NormalTok{(deltaE) }\OtherTok{\textless{}{-}} \FunctionTok{c}\NormalTok{(}\StringTok{"Original"}\NormalTok{, }\StringTok{"Política"}\NormalTok{, }\StringTok{"Diferencia"}\NormalTok{, }\StringTok{"Porcentual"}\NormalTok{)}

\CommentTok{\# Y si queremos el detalle}
\NormalTok{E1 }\OtherTok{\textless{}{-}}\NormalTok{ EC }\SpecialCharTok{\%*\%} \FunctionTok{diag}\NormalTok{(}\FunctionTok{as.vector}\NormalTok{(L }\SpecialCharTok{\%*\%}\NormalTok{ f1))}
\FunctionTok{colnames}\NormalTok{(E1) }\OtherTok{\textless{}{-}} \FunctionTok{colnames}\NormalTok{(E)}

\CommentTok{\# =============================================================================}
\CommentTok{\# Bioeconomía}

\CommentTok{\# Clasificación cruzada bioeconomía AECR}
\NormalTok{clasifs\_bioeconomia }\OtherTok{\textless{}{-}} \FunctionTok{read.xlsx}\NormalTok{(}\StringTok{"clasificacion\_cruzada\_bioeconomia.xlsx"}\NormalTok{, }
                                            \AttributeTok{sheet =} \StringTok{"datos"}\NormalTok{, }
                                            \AttributeTok{rows=} \FunctionTok{c}\NormalTok{(}\DecValTok{1}\SpecialCharTok{:}\DecValTok{137}\NormalTok{), }
                                            \AttributeTok{cols =} \FunctionTok{c}\NormalTok{(}\DecValTok{1}\SpecialCharTok{:}\DecValTok{6}\NormalTok{), }
                                            \AttributeTok{skipEmptyRows =} \ConstantTok{FALSE}\NormalTok{, }
                                            \AttributeTok{colNames =}\ConstantTok{TRUE}\NormalTok{, }
                                            \AttributeTok{rowNames =} \ConstantTok{FALSE}
\NormalTok{)  }\CommentTok{\# \textless{}{-}{-} Fin del paréntesis}





\CommentTok{\# =============================================================================}
\CommentTok{\# Excel}

\FunctionTok{write.xlsx}\NormalTok{(}\FunctionTok{as.data.frame}\NormalTok{(}\FunctionTok{colnames}\NormalTok{(Z)),}
                         \StringTok{"nombres\_Z.xlsx"}\NormalTok{,}
                         \AttributeTok{sheetName =} \StringTok{"datos"}\NormalTok{,}
                         \AttributeTok{startRow =} \DecValTok{5}\NormalTok{,}
                         \AttributeTok{startCol =} \DecValTok{1}\NormalTok{,}
                         \AttributeTok{asTable =} \ConstantTok{FALSE}\NormalTok{,}
                         \AttributeTok{colNames =} \ConstantTok{TRUE}\NormalTok{,}
                         \AttributeTok{rowNames =} \ConstantTok{TRUE}\NormalTok{,}
                         \AttributeTok{overwrite =} \ConstantTok{TRUE}\NormalTok{)}

\CommentTok{\# Cambios en datos ambientales}
\FunctionTok{write.xlsx}\NormalTok{( }\FunctionTok{as.data.frame}\NormalTok{(deltaE) , }
            \StringTok{"datos\_ambientales.xlsx"}\NormalTok{,}
            \AttributeTok{sheetName=} \StringTok{"datos"}\NormalTok{,}
            \AttributeTok{startRow =} \DecValTok{5}\NormalTok{,}
            \AttributeTok{startCol =} \DecValTok{1}\NormalTok{,}
            \AttributeTok{asTable =} \ConstantTok{FALSE}\NormalTok{, }
            \AttributeTok{colNames =} \ConstantTok{TRUE}\NormalTok{, }
            \AttributeTok{rowNames =} \ConstantTok{TRUE}\NormalTok{, }
            \AttributeTok{overwrite =} \ConstantTok{TRUE}
\NormalTok{            )}

\CommentTok{\# Gráfico}
\FunctionTok{heatmap}\NormalTok{(E1, }\AttributeTok{Colv =} \ConstantTok{NA}\NormalTok{, }\AttributeTok{Rowv =} \ConstantTok{NA}\NormalTok{)}
\end{Highlighting}
\end{Shaded}

\includegraphics{modulo_03_files/figure-latex/unnamed-chunk-1-1.pdf}

\begin{Shaded}
\begin{Highlighting}[]
\CommentTok{\# \# =============================================================================}
\CommentTok{\# \# Bloopers}
\CommentTok{\# }
\CommentTok{\# \# Inicialmente creímos que la matriz Z tenía un componente importado}
\CommentTok{\# \# y un componente nacional. Ese no es el caso, solo está dividida por}
\CommentTok{\# \# el tipo de control "nacional" o "extranjero" de la producción nacional.}
\CommentTok{\# \# Gracias a Johnny Aguilar por la aclaración.}
\CommentTok{\# }
\CommentTok{\# \# Dejo la solución para extraer filas y columnas con índices, solamente}
\CommentTok{\# \# porque es una buena ilustración de algo que se utiliza continuamente}
\CommentTok{\# \# en el trabajo con este tipo de datos.}
\CommentTok{\# }
\CommentTok{\# \# Obtenemos nuestros códigos de actividad y nombres}
\CommentTok{\# cods \textless{}{-} as.data.frame(}
\CommentTok{\#   read.xlsx(}
\CommentTok{\#     "MIP{-}AE{-}AE{-}017{-}CR.xlsx",}
\CommentTok{\#     sheet = "clasificaciones", }
\CommentTok{\#     rows= c(5:141), }
\CommentTok{\#     cols = c(1:5), }
\CommentTok{\#     skipEmptyRows = FALSE, }
\CommentTok{\#     colNames = TRUE, }
\CommentTok{\#     rowNames = FALSE}
\CommentTok{\#   )}
\CommentTok{\# )\# \textless{}{-}{-} Fin del paréntesis}
\CommentTok{\# }
\CommentTok{\# \# componente doméstico}
\CommentTok{\# local \textless{}{-} c(2,4,6,8,10,12,14,16,18,20,22,24,26,28,30, }
\CommentTok{\#            32,34,36,38,40,42,44,46,48,50,52,54,56,58,60, }
\CommentTok{\#            62,64,66,68,70,71,73,75,77,78,80,81,83,85,87,88, }
\CommentTok{\#            90,92,94,95,97,98,99,100,101,103,105,107,108,110, }
\CommentTok{\#            112,114,115,117,119,120,121,123,125,127,129,131, }
\CommentTok{\#            133,135,137,139,141,143,145,147,149,151,153,155, }
\CommentTok{\#            157,159,161,162,164,166,167,169,171,173,174,176, }
\CommentTok{\#            178,180,182,184,186,188,190,192,194,196,197,199, }
\CommentTok{\#            200,202,204,206,208,210,212,214,216,218,219,221, }
\CommentTok{\#            223,225,227,229,231,233,235,237,239,241,243,245, }
\CommentTok{\#            247,249,251,253)}
\CommentTok{\# }
\CommentTok{\# \# Si el BCR hubiera dejado (aunque en ceros) el componente importado}
\CommentTok{\# \# de los sectores que solamente tienen componente local, los indices}
\CommentTok{\# \# serían más fáciles de construir con:}
\CommentTok{\# \# seq( from=2, to= 254, by=2 ) para lo doméstico }
\CommentTok{\# }
\CommentTok{\# \# Aquí utilizamos el método del índice en R para obtener nuestros elementos}
\CommentTok{\# }
\CommentTok{\# \# Compras entre actividades económicas domésticas (consumo intermedio)}
\CommentTok{\# Z \textless{}{-} Z\_cruda[  local  ,  local   ]}
\CommentTok{\# }
\CommentTok{\# \# Compras de producto importado de las actividades domésticas (importaciones)}
\CommentTok{\# M \textless{}{-} Z\_cruda[{-}local,local]}
\CommentTok{\# }
\CommentTok{\# \# Compras de Actividades económicas no{-}domésticas a las actividades locales}
\CommentTok{\# \# (exportaciones)}
\CommentTok{\# X \textless{}{-} Z\_cruda[local,{-}local]}
\CommentTok{\# }
\CommentTok{\# \# Compras de Actividades no{-}domésticas a no{-}domésticas?}
\CommentTok{\# XX \textless{}{-} Z\_cruda[{-}local,{-}local]}
\end{Highlighting}
\end{Shaded}


\end{document}
